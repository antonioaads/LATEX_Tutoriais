% ------------------------------------------------------------------------
% ------------------------------------------------------------------------
% relaTeX - Modelo de relatório da equipe AeroDesign ITA, responsável: saulo
%
% Arquivos logo_equipe.pdf , logo_rodape.pdf e logo_SaeBrasil.pdf são auto-explicativos né?
%           [vai na capa]      [no rodapé]      [canto da capa]
%
% Usamos e abusamos de:
% abnTeX2: Modelo de Relatório Técnico/Acadêmico em conformidade com 
% ABNT NBR 10719:2011 Informação e documentação - Relatório técnico e/ou
% científico - Apresentação
%
% ------------------------------------------------------------------------ 
% ------------------------------------------------------------------------

\documentclass[
	% -- opções da classe memoir --
	12pt,				% tamanho da fonte
	openany,			% para impressão em verso e anverso. Oposto a oneside
	a4paper,			% tamanho do papel. 
	% -- opções da classe abntex2 --
	%chapter=TITLE,		% títulos de capítulos convertidos em letras maiúsculas
	%section=TITLE,		% títulos de seções convertidos em letras maiúsculas
	%subsection=TITLE,	% títulos de subseções convertidos em letras maiúsculas
	%subsubsection=TITLE,% títulos de subsubseções convertidos em letras maiúsculas
	% -- opções do pacote babel --
	english,			% idioma adicional para hifenização
	french,				% idioma adicional para hifenização
	spanish,			% idioma adicional para hifenização
	brazil,				% o último idioma é o principal do documento
	]{abntex2}


% ---
% PACOTES
% ---

% ---
% Pacotes fundamentais 
% ---
\usepackage{lmodern}			% Usa a fonte Latin Modern
\usepackage[T1]{fontenc}		% Selecao de codigos de fonte.
\usepackage[utf8]{inputenc}		% Codificacao do documento (conversão automática dos acentos)
\usepackage{indentfirst}		% Indenta o primeiro parágrafo de cada seção.
\usepackage{color}				% Controle das cores
\usepackage{graphicx}			% Inclusão de gráficos
\usepackage{microtype} 			% para melhorias de justificação
\usepackage{pstricks}
\graphicspath{{../IMG/}}
%##############
\usepackage{tabu}				% usar ambiente tabu no lugar de tabular
\setlength{\tabulinesep}{1.2mm} % Espaçando itens nas celulas da tabela
%##############

% ---
% Pacotes de citações
% ---
\usepackage[brazilian,hyperpageref]{backref}	 % Paginas com as citações na bibl
\usepackage{abntex2cite}	% Citações padrão ABNT

% --- 
% CONFIGURAÇÕES DE PACOTES
% --- 

% ---
% Configurações do pacote backref
% Usado sem a opção hyperpageref de backref
\renewcommand{\backrefpagesname}{Citado na(s) página(s):~}
% Texto padrão antes do número das páginas
\renewcommand{\backref}{}
% Define os textos da citação
\renewcommand*{\backrefalt}[4]{
	\ifcase #1 %
		Nenhuma citação no texto.%
	\or
		Citado na página #2.%
	\else
		Citado #1 vezes nas páginas #2.%
	\fi}%
% ---

% -------------------------------------------
% -------------------------------------------
% Informações de dados para CAPA e FOLHA DE ROSTO
% Informar o numero da equipe 
% O título é o nome da equipe!
% Colocar todos os membros como autores, pode colocar \at email se quiser..
% ---
\titulo{Relatório de aerodinamica}
\autor{AUTOR}
\local{Brasil}
\data{2015, v-1.0}
\instituicao{Instituto Tecnológico de Aeronáutica}
\orientador{Kleine}
\tipotrabalho{Relatório técnico}

% Configurações de aparência do PDF final

% alterando o aspecto da cor azul
\definecolor{blue}{RGB}{41,5,195}

     
% informações do PDF
\makeatletter
\hypersetup{
     	%pagebackref=true,
		pdftitle={\@title}, 
		pdfauthor={\@author},
    	pdfsubject={\imprimirpreambulo},
	    pdfcreator={LaTeX with abnTeX2},
		pdfkeywords={AeroDesign}{SAE Brasil}{abntex}{abntex2}{relatório técnico}, 
		colorlinks=true,       		% false: boxed links; true: colored links
    	linkcolor=red,          	% color of internal links
    	citecolor=blue,        		% color of links to bibliography
    	filecolor=magenta,      		% color of file links
		urlcolor=blue,
		bookmarksdepth=4
}
\makeatother
% --- 

% --- 
% Espaçamentos entre linhas e parágrafos 
% --- 

% O tamanho do parágrafo é dado por:
\setlength{\parindent}{1.3cm}

% Controle do espaçamento entre um parágrafo e outro:
\setlength{\parskip}{0.2cm}  % tente também \onelineskip

% Espaçamento de itens dentro da celula
\setlength{\tabulinesep}{1.2mm}
% ---
% compila o indice
% ---
\makeindex
% ---

% ----
% Início do documento
% ----
\begin{document}

% Seleciona o idioma do documento (conforme pacotes do babel)
%\selectlanguage{english}
\selectlanguage{brazil}

% Retira espaço extra obsoleto entre as frases.
\frenchspacing 

% ----------------------------------------------------------
% ELEMENTOS TEXTUAIS
% ----------------------------------------------------------
\textual
\chapter{Projeto conceitual}\label{prj.cap}

\postextual

\chapter*{Simbolos}
\begin{itemize}

\item[$\alpha$] Ângulo de ataque
\item[$\alpha _s$] Ângulo de ataque de estol
\item[$\beta$] Ângulo de derrapagem
\item[$\varphi$] Ângulo de inclinação lateral
\item[$\theta$] Ângulo de atitude
\item[$\psi$] Ângulo de proa
\item[$C_{m_{\alpha}}$] Derivada do coeficiente de momento de arfagem com relação ao ângulo de ataque
\item[$C_{L_{\alpha}}$] Derivada do coeficiente de sustentação com relação ao ângulo de ataque
\item[$C_{l_{\beta}}$] Derivada do coeficiente de momento de rolamento com relação ao ângulo de derrapagem
\item[$C_{n_{\beta}}$] Derivada do coeficiente de momento de guinada com relação ao ângulo de derrapagem
\item[$\delta _a$] Deflexão de aileron
\item[$\delta _p$] Deflexão de profundor
\item[$\delta _l$] Deflexão de leme
\item[$\delta _m$] Comando de manete
\item[$C_L$] Coeficiente de sustentação
\item[$C_M$] Coeficiente de momento de arfagem
\item[$C_{HT}$] Coeficiente de volume de cauda horizontal
\item[$C_{VT}$] Coeficiente de volume de cauda vertical
\item[$c_l$] Coeficiente de sustentação de perfil
\item[$V$] Velocidade
\item[$V_{stall}$] Velocidade de estol
\item[$V_{man}$] Velocidade de manobra
\item[$V_{cruz}$] Velocidade de cruzeiro
\item[$V_{r}$] Velocidade de rotação para decolagem em pista
\item[$W$] Peso da aeronave
\item[$T$] Tração
\item[$L$] Sustentação
\item[$D$] Arrasto
\item[$\mu$] Coeficiente de atrito
\item[$\alpha _F$] Ângulo do motor com relação ao eixo longitudinal
\item[$g$] Aceleração da gravidade
\item[$n_z$] Fator de carga vertical
\item[$CP$] Centro de pressão
\item[$CG$] Centro de gravidade
\item[$ME$] Margem estática
\item[$R$] Raio de curva
\item[$MTOW$] \textit{Maximum Weight take off}

\end{itemize}
\bibliography{../aero}


\end{document}
